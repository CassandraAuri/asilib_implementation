\documentclass[draft]{agujournal2019}
\usepackage{url}
\usepackage{lineno}
\usepackage{soul}

\linenumbers
\draftfalse
\journalname{Earth and Space Science}


\begin{document}

\title{AuroraX and aurorax-asilib: a user-friendly auroral all-sky imager analysis framework}

\authors{M. Shumko\affil{1, 2}, B. Gallardo-Lacourt\affil{1,2}, A.J. Halford\affil{1}, E. Donovan\affil{3}, K.R. Murphy, E.L. Spanswick\affil{3}, D. Chaddock\affil{3}, and I. Thompson}

\affiliation{1}{NASA's Goddard Space Flight Center, Greenbelt, Maryland, USA}
\affiliation{2}{Universities Space Research Association, Columbia, Maryland, USA}
\affiliation{3}{University of Calgary, Calgary, Alberta, Canada}

\correspondingauthor{M. Shumko}{msshumko@gmail.com}


\begin{keypoints}
\item AuroraX is a seamless online interface to visualize the aurora
\item aurorax-asilib is a Python package for the detailed analysis of auroral all-sky imager data 
\item Together, these tools lower the barriers to entry for scientists who want to study the aurora 
\end{keypoints}


\begin{abstract}
Abstract
\end{abstract}


\section*{Plain Language Summary}
\noindent


\section{Introduction}\label{intro}
\textcolor{blue}{
      OUTLINE
      \begin{itemize}
            \item Brief history of ASIs and ASI arrays. Talk about why THEMIS ASI exists. Discuss CANOPUS?
            \item With these ASI arrays, the volume of data is immense.
            \item Why this software? Aurora ASI data formats very greatly, each with their own caveats. Our goal is to remove the need for scientists needing to write duplicate code to use these popular missions. AS a result, this will enable scientists to dive right into the science and not need to know the details of data management (downloading and loading data, as well as applying routine data processing steps)   
      \end{itemize}
}


\section{Design Philosophy}
\textcolor{blue}{
      OUTLINE
      \begin{itemize}
            \item The primary design philosophy is to offer a relatively small set of functionality that is useful for most researchers studying the aurora. We strived to strike a balance between a complex and a user-friendly tool. 
            \item Online keogram and conjunction interface accessible anywhere with internet connection.
            \item Comprehensive ASI data analysis functionality on a PC.
      \end{itemize}
}

\section{AuroraX}\label{aurorax}
\textcolor{blue}{
      OUTLINE
      \begin{itemize}
            \item What is it?
            \item A highly optimized conjunction search
            \item On-demand keograms
            \item Virtual Observatory
      \end{itemize}
}

\section{aurorax-asilib}\label{aurorax-asilib}
\textcolor{blue}{
      OUTLINE
      \begin{itemize}
            \item What is it?
            \item Plug-in based architecture
            \item Handles the downloading and loading of ASI images. Ultimately, ASI image files consists of time stamps and images, so the load data is equally as simple.
            \item Similarly with skymap calibration files.
            \item If a file is already downloaded, you do not need an internet connection to work with the data.
            \item 
      \end{itemize}
}
\subsection{Download and load ASI image and skymap data}

\subsection{Plotting single images}

\subsection{Creating ASI movies}

\subsection{ASI analysis tools}

\subsection{An example: a satellite-ASI conjunction}

\section{Quality Assurance}
\textcolor{blue}{
      OUTLINE
      \begin{itemize}
            \item asilib on GitHub. unit and integration tests run automatically before every release.
            \item THEMIS and REGO data formats are set and won't change.
      \end{itemize}
}

\section{Conclusion}

\acknowledgments
We are thankful for the engineers and scientists who made AuroraX, THEMIS ASI, and REGO ASI projects possible. M. Shumko and B. Gallardo-Lacourt acknowledge the support provided by the NASA Postdoctoral Program at the NASA’s Goddard Space Flight Center, administered by Universities Space Research Association under contract with NASA. The THEMIS and REGO ASI data is available from \url{https://data.phys.ucalgary.ca/}.

% \bibliography{refs.bib}

\end{document}